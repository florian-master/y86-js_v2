
\documentclass[french]{article}
 
\usepackage[utf8]{inputenc}
\usepackage[T1]{fontenc}
\usepackage{babel}

\usepackage[colorlinks=true,linkcolor=black,urlcolor=blue]{hyperref}

\title{Compte rendu de TD - Environnement d'exécution y86+HCL}

\author{
    BANDET Alexis \\
    \texttt{alexis.bandet@u-bordeaux.fr} \\
    GAISSET Valentin \\
    \texttt{valentin.gaisset@etu.u-bordeaux.fr} \\
    GUISSET Romain \\
    \texttt{romain.guisset@etu.u-bordeaux.fr} \\
    SIMBA Florian \\
    \texttt{florian.simba@u-bordeaux.fr} \\
}

\date{23 Janvier 2020}

\begin{document}

\maketitle

\section{Introduction}

Première séance de TD. Présentation de notre compréhension du sujet et détails de la rencontre avec le client. 

\section{Éléments à corriger}

\begin{enumerate}
    \item Détailler les besoins :\\
    Cela inclura aussi de pouvoir mesurer à quel point chacun des besoins aura été répondu. Le cahier des besoins doit être assez précis et détaillé pour qu'il suffise à une équipe de développement à mettre en oeuvre le projet sans pré-acquis.
    Il faudra par ailleurs lister ce qui est analysé (dans ce qui sera produit) pour certains besoins. Enfin il faut que le cahier des besoins puisse servir à vérifier si un besoin spécifique est résolue ou non à la fin du projet. 
    \\
    \item Besoins spécifiquement exprimés par les clients à inclure dans le Cahier des Besoins :
    \begin{itemize}
        \item Le code sera sous licence GPLv3
        
        \item Le répertoire \textit{git} où s'effectuera le développement se trouvera en dehors de \textit{savane},à savoir sur le \textit{GitLab} de l'INRIA. Il sera toute fois nécessaire de pousser les versions stables de l'application sur \textit{savane} pour permettre un suivit de notre projet par les responsables de l'UE PdP
        
        \item Il faut aussi préciser notre besoin d'analyse de projets pré-existant. C'est un besoin non fonctionnel important.
    \end{itemize}
    
    \item Croquis de l'interface utilisateur :\\
    Ils sont à réaliser au plus tôt, même s'ils ne reflètent pas exactement le résultat final de l'application.
    
\end{enumerate}

% TODO

\end{document}